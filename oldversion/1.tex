\documentclass[UTF8,a4paper,fontset=none]{ctexart}
\usepackage[a4paper, left = 2cm, right = 2cm,top=25mm,bottom=25mm]{geometry}
\usepackage{ctex}
\usepackage{graphicx}
\usepackage{fontspec}
\usepackage{mathtools}
\usepackage{gensymb}
\usepackage{float}
\usepackage{xcolor}
\usepackage{fancyhdr}
\usepackage{amssymb}
\usepackage{indentfirst}
\usepackage{titlesec}
\setlength{\parindent}{2em}
\newcommand{\heading}{测试报告}%页眉在此修改
\pagestyle{fancy}
\fancyhead[]{}
{\renewcommand{\headrulewidth}{0.5pt}}
\fancyhead[L]{\fangsong \color[rgb]{0.15234375,0.19140625,0.4921875}\heading}
\fancyhead[R]{\thepage}
\fancypagestyle{plain}{
\fancyhf{} % clear all header and footer fields
{\renewcommand{\headrulewidth}{0.5pt}}
\fancyhead[L]{\fangsong \color[rgb]{0.15234375,0.19140625,0.4921875}\heading}
\fancyhead[R]{\thepage}
\fancyfoot[L]{\fangsong \color[rgb]{0.15234375,0.19140625,0.4921875} 上交日期:\ \color{black}\today}
}
\title{
    {\huge {\color[rgb]{0.15234375,0.19140625,0.4921875}\heiti 测试报告}}\\%中文标题
    {\huge \color[rgb]{0.15234375,0.19140625,0.4921875}\fontspec{Arial} Test Report}\\%英文标题
}
\author{
    \fangsong 余笑轩\\%姓名
    \fangsong ( 化学与分子工程学院\qquad 1900011816 )%学院和学号    
}
\date{}
\ctexset{fontset=adobe,section={format=\raggedright\heiti\zihao{-2}\fontspec{Arial}\color[rgb]{0.15234375,0.19140625,0.4921875},numbering=false},subsection={format=\raggedright\heiti\zihao{-3}\fontspec{Arial}\color[rgb]{0.15234375,0.19140625,0.4921875},numbering=false},subsubsection={format=\raggedright\heiti\zihao{-4}\fontspec{Arial}\color[rgb]{0.15234375,0.19140625,0.4921875},numbering=false}}
\makeatletter
\def\@maketitle{%
  \newpage
  \begin{center}%
  \let \footnote \thanks
    {\LARGE \@title \par}%
    \vskip 1em%
    {\large
      \begin{tabular}[t]{c}%
        \@author
      \end{tabular}\par}%
    \vskip 0em%
    {\large \@date}%
  \end{center}%
  \par
  \vskip 1.5em}
\makeatother
%正文从这里开始
\begin{document}
    \zihao 4
    \titleformat{\section}{\raggedright\heiti\zihao{-2}\fontspec{Arial}\color[rgb]{0.15234375,0.19140625,0.4921875}}{$\blacksquare$}{1em}{}{}
    \titleformat{\subsection}{\raggedright\heiti\zihao{-3}\fontspec{Arial}\color[rgb]{0.15234375,0.19140625,0.4921875}}{}{28pt}{}{}
    \titleformat{\subsubsection}{\raggedright\heiti\zihao{4}\fontspec{Arial}\color[rgb]{0.15234375,0.19140625,0.4921875}}{}{28pt}{}{}
    \maketitle
    \section{Section}
    \subsection{Subsection}
    \subsubsection{Subsubsection}
    古代彩绘的装饰程度和绘画技巧是考古学的重要研究对象,因为除了一些保存完好的残片外,大理石艺术品上附着的色彩只剩下显微级别的痕迹。X射线荧光光谱能做到现场、无损分析古物,而为了进一步研究古物上颜料残余的分布情况,对分析方法提出了角度限制、高分辨率、高对比度、穿透性等更高要求,只有X射线成像技术能够全部满足。然而这一方法的主要困难在于,准确的成像需要对样品的3D模型来支撑,博物馆环境又使得建模需要在较短时间内现场完成,并要求装备简便,测量过程不接触和破坏样品。摄影测量法适用于这一场景,而基本参数计算又能对测量时不同几何位置带来的信号强度变化进行校正。作者创新地将两种技术结合在一起,用于对建造于公元前525年左右的,德尔斐(希腊)的锡弗诺斯人宝库的雕刻横饰带进行研究。这一遗迹坍塌后,直至1893年被重新发掘出土。除了少量清洁之外,雕塑仍包裹着大量的沉积物,且仍有颜料留存,其中一部分保存在德尔斐考古博物馆中。本文研究的区域是其中阿喀琉斯手持的美杜莎纹饰的盾牌,Brinkmann于1980年在可见-红外照明下探明了雕带中的多种矿物颜料,却未在这一蛇妖的头上发现颜料的痕迹。于是本文通过XRF荧光成像技术使其大放异彩。
    \clearpage
    \section{section 2}
    古代彩绘的装饰程度和绘画技巧是考古学的重要研究对象,因为除了一些保存完好的残片外,大理石艺术品上附着的色彩只剩下显微级别的痕迹。X射线荧光光谱能做到现场、无损分析古物,而为了进一步研究古物上颜料残余的分布情况,对分析方法提出了角度限制、高分辨率、高对比度、穿透性等更高要求,只有X射线成像技术能够全部满足。然而这一方法的主要困难在于,准确的成像需要对样品的3D模型来支撑,博物馆环境又使得建模需要在较短时间内现场完成,并要求装备简便,测量过程不接触和破坏样品。摄影测量法适用于这一场景,而基本参数计算又能对测量时不同几何位置带来的信号强度变化进行校正。作者创新地将两种技术结合在一起,用于对建造于公元前525年左右的,德尔斐(希腊)的锡弗诺斯人宝库的雕刻横饰带进行研究。这一遗迹坍塌后,直至1893年被重新发掘出土。除了少量清洁之外,雕塑仍包裹着大量的沉积物,且仍有颜料留存,其中一部分保存在德尔斐考古博物馆中。本文研究的区域是其中阿喀琉斯手持的美杜莎纹饰的盾牌,Brinkmann于1980年在可见-红外照明下探明了雕带中的多种矿物颜料,却未在这一蛇妖的头上发现颜料的痕迹。于是本文通过XRF荧光成像技术使其大放异彩。
\end{document}